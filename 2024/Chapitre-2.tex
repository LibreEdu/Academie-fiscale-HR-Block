\chapter{Revenus d’emploi}
\section{Déclarer un revenu d’emploi}
Fédéral:
\ca
\begin{description}
	\item[10100] Revenus d'emploi (T4.14)
	\item[10120] Commissions incluses à la ligne 10100 (T4.42)\footnote{\label{inf}Informatif}
	\item[10130] Cotisations à un régime d'assurance-salaire (cf. ligne \href{https://www.canada.ca/fr/agence-revenu/services/impot/particuliers/sujets/tout-votre-declaration-revenus/declaration-revenus/remplir-declaration-revenus/revenu-personnel/ligne-10100-revenus-emploi.html}{10100})
	\item[10400] Autres revenus d'emploi
	\item[13000] Autres revenus
\end{description}

Québec:
\qc
\begin{description}
	\item[100] Commissions reçues (RL1.M)\textsuperscript{\,\ref{inf}}
	\item[101] Revenus d'emploi (RL1.A)
	\item[105] Correction des revenus d'emploi, si RL22
	\item[107] Autres revenus d'emploi
	\item[154] Autres revenus
\end{description}
\subsection{Feuillets T4 et relevés 1}
\href{https://www.canada.ca/fr/agence-revenu/services/formulaires-publications/formulaires/t4.html}{T4} -- \href{https://www.revenuquebec.ca/fr/services-en-ligne/formulaires-et-publications/details-courant/rl-1/}{RL1}
\subsubsection{Case 14 du T4 et case A du relevé 1}
Le plus souvent RL1.A $\ge$ T4.14: montants imposables au QC, mais pas au fédéral.
\subsubsection{Retenues à la source et autres montants inscrits sur les T4 et relevé 1}
\begin{itemize}
	\item Impôt fédéral (T4.22) et impôt Qc (RL1.E)
	\item Cotisations Régime des rentes du Québec (T4.17, RL1.B)
	\item Cotisations assurance emploi (T4.18, RL1.C)
	\item Cotisations Régime québécois d’assurance parentale (T4.55, RL1.H)
	\item Cotisations Régime de pension agréé (T4.20, RL1.D)
	\item Cotisations syndicales (T4.44, RL1.F)
	\item Dons de bienfaisance (T4.46, RL1.N)
\end{itemize}

Autres info obligatoires:
\begin{itemize}
	\item Gains assurables à l’assurance emploi (T4.24)
	\item Salaire admissible au Régime des rentes du Québec (T4.26, RL1.G)
	\item Salaire admissible au Régime québécois d’assurance parentale (T4.56, RL1.I)
	\item Facteur d’équivalence (T4.52)
\end{itemize}

Gains assurables (T4) = Salaires admissibles (RL1)

Avantages imposables au Québec et pas au fédéral : Régime privé d’assurance maladie (RL1.J) et Régime d’assurance interentreprises (RL1.P).
\[ A = J + K + L + P + V + W\]
\begin{description}
	\item[A] Revenus d’emploi
	\item[J] Régime privé d’assurance maladie 
	\item[K] Voyages (région éloignée) 
	\item[L] Autres avantages
	\item[P] Régime d’assurance interentreprises 
	\item[V] Nourriture et logement 
	\item[W] Véhicule à moteur 
\end{description}
\section{Régime privé d'assurance-maladie payé par l'employeur}
Les primes versées aux régimes privés d‘assurance maladie sont imposable au QC, mais pas au fédéral.
\subsection{Régime d'assurance interentreprises}
Régime d’assurance collective offert par plusieurs employeurs appartenant à un même secteur économique.
\subsection{Fédéral -- Autres revenus d'emploi}
Les primes versées par l’employeur à un régime d’assurance-vie collective
temporaire sont imposables au fédéral. Ce montant n'est pas inclus sur le T4. T4A.119, primes payées pour une police d'assurance-vie collective temporaire, reporté T1.10400.
\subsection{Québec -- Correction des revenus d’emploi (Ligne 105)}
Les primes versées par l’employeur à un régime d’assurance-vie et celles versées à un régime d’assurance maladie sont imposables au QC. Montant estimé dans RL1.P et inclut dans RL1.A.

L'administrateur émet un \href{https://www.revenuquebec.ca/fr/services-en-ligne/formulaires-et-publications/details-courant/rl-22/}{RL22}.

RL22.A $=$ montant réel de l'avantage.

RL22.B $=$ avantage relatif à l'assurance maladie

RL22.A $-$ RL22.B $=$ T4A.119

RL22.A $-$ RL1.P $=$ Correction des revenus d’emploi $=$ TP1.105 (grille 105)
\section{Compensation versée à un volontaire participant à des services d'urgence}
La première tranche de \numprint{1000}~\$ (fédéral) et \numprint{1235}~\$ (QC) de cette compensation est exonérée d’impôt. Le feuillet T4 et le relevé~1 sont émis seulement pour la partie du montant payé qui dépasse cette exemption.

Partie imposable: T4.14 et RL1.A

Partie non imposable: T4.87 et RL1.L2
