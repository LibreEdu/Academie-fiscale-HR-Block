\chapter{Déductions et crédits d'emploi}
\section{Introduction et objectifs}
\subsection{Introduction}
Déductions et crédits qui s'appliquent au revenu d'emploi. On retrouve les déductions dans deux parties des T1 et TP-1: le calcul du revenu net et le calcul du  revenu imposable. De plus, nous étudierons certains crédits d'impôt non remboursables reliés à l'emploi qui peuvent être réclamés à l'étape 5 de la T1, ainsi qu'à la page 3 de la TP-1.


\subsection{Objectifs}
\begin{itemize}
	\item Expliquer ce qu'on entend par déduction pour travailleur et comment la réclamer;
	\item Réclamer la déduction pour un régime de pension agréé;
	\item Réclamer la déduction fédérale et le crédit d'impôt non remboursable du Québec relativement aux cotisations syndicales, professionnelles et autres cotisations semblables;
	\item Réclamer les dépenses d'emploi liées au travail à domicile en raison de la COVID-19;
	\item Expliquer comment le contribuable qui déménage pour occuper un emploi dans un nouveau lieu de travail peut réclamer des frais de déménagement;
	\item Réclamer la déduction pour remboursement de salaires ou de prestations d'assurance salaire;
	\item Réclamer la déduction pour le personnel des Forces canadiennes et des forces policières;
	\item Réclamer la déduction pour options d'achat de titres pour les employés;
	\item Réclamer la déduction pour ristourne reçue d'une coopérative;
	\item Réclamer la déduction pour revenu d'emploi gagné sur un navire;
	\item Calculer la cotisation que doit verser un employé au Régime de rentes du Québec (RRQ) et déterminer les cotisations versées en trop;
	\item Réclamer une déduction pour la cotisation bonifiée au RRQ sur un revenu d'emploi; 
	\item Calculer la cotisation que doit payer un employé au Régime québécois d'assurance parentale et déterminer les cotisations versées en trop;
	\item Calculer la cotisation que doit verser un employé à l'assurance-emploi et déterminer les cotisations versées en trop;
	\item Réclamer le montant canadien pour emploi;
	\item Réclamer le crédit d'impôt non remboursable pour pompiers volontaires ou volontaires participant à des activités de recherche et de sauvetage;
	\item Appliquer les règles régissant qui doit déclarer un revenu rétroactif de la Prestation universelle pour la garde d'enfants (PUGE) fédérale; et
	\item Réclamer le crédit d'impôt non remboursable pour les nouveaux diplômés travaillant dans une région ressource éloignée.
\end{itemize}


\subsection{Sujets du chapitre 3}
\begin{itemize}
	\item Déductions
	\item Déductions du revenu d'emploi
	\item Revenu net et imposable
	\item RRQ, RQAP et AE
	\item Crédits non remboursables pour revenu d'emploi
\end{itemize}



\section{Distinction entre déductions et crédits d'impôt}
\begin{intro}
	La première partie de ce chapitre concerne les déductions les plus communes qui sont reliées à l'emploi. L'application de ces déductions conduit au calcul du revenu net.
	
	La deuxième partie de ce chapitre concerne les déductions plus spécifiques qui sont aussi reliées à l'emploi. Celles-ci ne concernent que quelques contribuables. L'application de ces déductions fiscales spécifiques, conduit au calcul du revenu imposable.
	
	Enfin, la troisième partie de ce chapitre concerne les crédits d'impôt non remboursables qui sont reliés à l'emploi. Pour cette partie, nous verrons ceux pour le fédéral en particulier.
	
	Il est important de comprendre les différences entre les déductions et les deux types de crédits (remboursables et non remboursables) et comment ils sont utilisés dans le calcul de l'impôt sur le revenu.
\end{intro}
\begin{note}
	Les trousses d'impôt comprennent des guides qui fournissent des explications pour remplir chaque déclaration de revenus. Dans la transition vers le monde numérique, le guide fédéral a constaté une réduction des explications ligne par ligne. Si vous ne trouvez pas d'explication dans le guide, utilisez le champ de recherche sur le site Web de l'ARC et entrez le numéro de ligne.
	
	Cliquez sur le bouton ci-dessous pour afficher une liste complète de chaque ligne dans les étapes 3 à 6.
	
	\href{https://www.canada.ca/fr/agence-revenu/services/impot/particuliers/sujets/tout-votre-declaration-revenus/declaration-revenus/remplir-declaration-revenus/deductions-credits-depenses/toutes-deductions-tous-credits-toutes-depenses.html}{Toutes les déductions, tous les crédits et toutes les dépenses}
\end{note}


\subsection{Déduction}
Une déduction réduit le revenu assujetti à l'impôt et, par conséquent, réduit en fin de compte l'impôt qu'une personne doit payer.

Il existe deux séries de déductions. Le premier ensemble réduit le revenu total dans le calcul du revenu net. Il se trouve à l'étape 3 - Revenu net de la T1 et à la section \og Revenu net \fg{} de la TP-1. 

Le deuxième ensemble de déductions réduit le revenu net dans le calcul du revenu imposable. Il se trouve à l'étape 4 - Revenu imposable de la T1 et à la partie \og Revenu imposable \fg{} de la TP-1.


\subsection{Crédits d'impôt non remboursables}
En ce qui concerne les crédits d'impôt non remboursables, ils servent à réduire l'impôt à payer plutôt que le revenu. 

Les crédits d'impôt ont la même valeur pour tous puisqu'ils sont accordés à un taux fixe pour tous les contribuables, peu importe leur revenu et leur taux marginal d'imposition. 

Les crédits d'impôt non remboursables sont déterminés pour la déclaration fédérale dans la partie des crédits d'impôt non remboursables à la section 5 de la T1, puis convertis au taux de 15~\%. D'autres crédits d'impôt non remboursables ont des taux différents, qui seront calculés plus tard, notamment dans la Partie C - Impôt fédéral net de l'Étape 5 de la T1.

Au Québec, les crédits non remboursables peuvent être réclamés dans deux parties, les crédits d'impôt non remboursables et les revenus et cotisations à la page 3 du TP-1. Pour de nombreux crédits, le taux est de 14~\%, cependant il existe une fourchette de taux plus large que le fédéral variant entre 8~\% et
30~\% selon les crédits.

Si la valeur totale des crédits non remboursables est égale ou supérieure à l'impôt à payer, le contribuable ne tire aucun avantage du surplus. C'est pourquoi ils sont appelés non remboursables.

Au Québec, cependant, nous verrons dans un chapitre ultérieur comment un conjoint ayant un excédent de crédits non remboursables peut les transférer à sa conjointe si elle est en mesure d'utiliser l'excédent. 


\subsection{Crédits d'impôt remboursables}
Les crédits remboursables sont également utilisés pour réduire l'impôt à payer plutôt que le revenu. 

Toutefois, si un surplus est créé lorsque ces crédits sont appliqués à l'impôt à payer, ce surplus est retourné au contribuable sous forme de remboursement. 

Ceux-ci sont réclamés à l'Étape 6 - \og Remboursement ou solde dû \fg{} de la T1 et \og Remboursement ou solde à payer \fg{} de la TP-1. 

Il existe deux types de crédits remboursables: 
\begin{itemize}
	\item Le premier type est des crédits pour des montants qu'un contribuable a déjà payés ou payés en trop. Le plus souvent, ces montants étaient déduits à la source; et
	\item Le deuxième type englobe les crédits d'impôt accordés par les gouvernements pour des programmes de redistribution des revenus ou de promotion de certains objectifs et politiques sociales.
\end{itemize}



\section{Calcul du revenu net}
\begin{intro}
	Le revenu net est déterminé en appliquant les déductions admissibles au montant du Revenu total. 
	
	Dans cette partie, vous découvrirez les déductions les plus courantes liées à l'emploi.
\end{intro}
Après avoir déterminé son revenu total, un contribuable doit identifier toutes les déductions auxquelles il est admissible.

Dans cette 1ère partie de chapitre, nous étudions les déductions qui sont reliées à un revenu d'emploi. Elles comprennent, entre autres:
\begin{itemize}
	\item La déduction pour travailleur, au Québec seulement;
	\item La déduction pour un régime de pension agréé;
	\item Les cotisations annuelles syndicales et professionnelles (sujet spécial);
	\item Déduction pour la cotisation bonifiée au RRQ sur un revenu d'emploi; et
	\item Les frais de déménagement.
\end{itemize}

Au fédéral, on déduit le remboursement des prestations de programmes sociaux à la ligne 23500.

Le revenu net est utilisé pour déterminer l'admissibilité du contribuable à certains crédits d'impôt et prestations sociales.

\begin{note}
	Sur les des déclarations fédérale et du Québec, le contribuable additionne toutes les déductions auxquelles il a droit, puis soustrait simplement le résultat du revenu total pour obtenir le revenu net.
	
	Étant donné que les déductions sur la déclaration fédérale et celle du Québec ne sont pas toujours les mêmes, il est tout à fait normal que les montants du revenu net soient différents.
\end{note}



\section{Déduction pour travailleurs}
\begin{intro}
	Le gouvernement du Québec offre à tous les contribuables ayant un revenu d'emploi une compensation financière afin d'assumer les petites dépenses quotidiennes reliées à l'emploi.
\end{intro}

Au Québec, sur la TP-1, le contribuable ayant un revenu d'emploi peut bénéficier d'une déduction pour travailleur correspondant à 6~\% du revenu de travail admissible, jusqu'à concurrence de \numprint{1315}~\$ pour l'année d'imposition 2023. 

Cette déduction se calcule sur la grille de calcul 201, présentée au tableau 3-1. Elle est réclamée à la ligne 201, réduisant ainsi le revenu net.

Les revenus de travail donnant droit à la déduction pour travailleurs comprennent:
\begin{itemize}
	\item Revenus d'emploi des lignes 101, 107 et si positif, ligne 105;
	\item Le montant net des subventions de recherche;
	\item Prestations du Programme de protection des salariés (sera discuté au chapitre 7);
	\item Montants reçus dans le cadre d'un projet d'incitation au travail.
\end{itemize}
\begin{note}
	Le revenu d'emploi composé uniquement d'avantages imposables obtenus en raison d'un emploi antérieur, ne constitue pas un revenu admissible aux fins de cette déduction. Le montant inscrit à la case 211 du Relevé 1 doit être soustrait du montant inscrit à la ligne 101.
\end{note}
Sur la déclaration T1, l'équivalent de cette déduction est le Montant canadien pour emploi, qui est un crédit non remboursable.
\begin{note}
	Lorsque les revenus d'emploi sont supérieurs à \numprint{21916,67}~\$, la déduction maximum est de \numprint{1315}~\$.
\end{note}



\section{\acrfull{rpa}}
\begin{intro}
	Un régime de pension agréé est un régime de pension enregistré auprès de l'ARC conformément à la Loi de l'impôt sur le revenu. Le régime est mis en place par un employeur pour fournir un revenu de pension à ses employés au moment de leur retraite
\end{intro}
Les cotisations qui peuvent être versées à un RPA comprennent les cotisations pour services courants et celles pour services passés:
\begin{itemize}
	\item Service courant en 2023
	\item Service antérieur de 1990 à 2022
	\item Service antérieur pour 1989 et les années précédentes.
	\item 
\end{itemize}


\subsection{Déduction pour cotisations pour services courants}
Le montant retenu par l'employeur est indiqué aux cases 20 du feuillet T4 et D du relevé 1. Il peut également figurer sur un reçu émis par le syndicat du travailleur.

Une déduction peut être réclamée aux lignes 20700 de la déclaration T1 et 205 de la TP1.

Si un montant paraît à la case 52 du feuillet T4 et qu'aucun montant n'est inscrit aux cases 20 du T4 et D du relevé 1, cela signifie que l'employeur a versé toutes les cotisations au régime pour le bénéfice de l'employé.


\subsection{Cotisations pour services passés entre 1990 et 2022}
Au fédéral, les cotisations pour services passés rendus entre 1990 et 2021 sont généralement inscrites à la case 20 du T4. Lorsque la relation employeur-employé n'existe plus, les cotisations versées sont inscrites à la case 032 du feuillet T4A par l'administrateur du régime.

Au Québec,le montant de la cotisation pour services passés rendus après 1989 est indiqué à la case D du relevé 1.

Les cotisations pour services passés rendus après 1989 sont entièrement déductibles seulement dans l'année où elles sont versées. Cela signifie que les cotisations pour services passés versées en 2022 doivent être déduites de la déclaration de revenus de 2022. Il n'existe aucun mécanisme permettant de demander la déduction pour les années suivantes.

Tout comme les cotisations pour services courants, elles sont réclamées aux lignes 20700 de la T1 et 205 de la TP1.

La déduction réclamée au Québec ne peut pas excéder celle demandée au fédéral.


\subsection{Cotisations pour services passés rendus avant 1989}
Les cotisations pour services passés rendus avant 1990 sont incluses à la case 20 du T4 et à la case D du relevé 1. Elles sont également inscrites à la case 74 ou 75 du T4 et à la case complémentaire D-2 ou D-3 du relevé 1.

S'il n'y a aucune inscription aux cases 74 et 75 du T4 ou aux cases D-2 et D-3 du relevé 1, alors le montant des cases 20 et D est entièrement déductible puisqu'il comprend uniquement des cotisations pour services courants ou pour services passés rendus après 1989.



\section{Exercice 3-1}
\setcounter{question}{0}
\begin{question}
	Quelle est la fonction des déductions?
\end{question}
Les déductions sont utilisées pour réduire le revenu.

Il existe deux séries de déductions. Le premier ensemble est utilisé pour réduire le revenu total afin de calculer le revenu net. Le second ensemble est utilisé pour réduire le revenu net afin de calculer le revenu imposable.

\begin{question}
	Quelle est la fonction des crédits d'impôt non remboursables?
\end{question}
Les crédits d'impôt non remboursables sont utilisés pour réduire l'impôt à payer.

\begin{question}
	Qu'est-ce qui différencie le traitement fiscal de la \og Déduction pour travailleur \fg{} de celui du \og Montant canadien pour emploi \fg{}? 
\end{question}
Au Québec, la déduction pour travailleurs est une déduction visant à réduire le revenu net du contribuable.

Le montant canadien pour emploi est un crédit d'impôt non remboursable permettant au contribuable de réduire son impôt à payer, lequel est établi à même son revenu imposable.

\begin{question}
	Durant l'année d'imposition 2023, en plus de sa cotisation pour son RPA de 460~\$ retenue sur son salaire, Barbara a également versé une cotisation de \numprint{6000}~\$ à son RPA pour des services passés rendus de 1994 à 1996.
\end{question}
\setcounter{sousQuestion}{0}
\begin{sousQuestion}
	Indiquez le montant qui sera inscrit aux cases 20 du T4 et D du relevé 1 émis par son employeur?
\end{sousQuestion}
\numprint{6460}~\$ (460~\$ + \numprint{6000}~\$)
\begin{sousQuestion}
	Est-ce que sa cotisation de \numprint{6000}~\$ pour services passés est entièrement déductible en 2023?  Quel montant peut-elle réclamer aux lignes 20700 de la T1 et 205 de la TP-1?
\end{sousQuestion}
Oui, ses cotisations pour services rendus après 1989 sont entièrement déductibles en 2024. Barbara peut réclamer \numprint{6400}~\$ (460~\$ + \numprint{6000}~\$) à la ligne 20700 de la T1 et à la ligne 205 de la TP-1.
\begin{sousQuestion}
	Au lieu de la déduire en 2023, Barbara peut-elle choisir de reporter sa cotisation de \numprint{6000}~\$ et réclamer la déduction dans une année subséquente?
\end{sousQuestion}
Non, si elle ne les réclame pas en 2023, elle ne pourra pas les réclamer dans une année subséquente.

Notez que les cotisations pour services passés antérieurs à 1990 sont assujetties à des règles différentes qui ne sont pas abordées dans ce cours.



\section{Régimes volontaires d'épargne-retraite (RVER) du Québec}
\begin{intro}
	Le but premier d'un régime volontaire d'épargne-retraite (RVER) est de donner accès à des régimes de retraite aux contribuables travailleurs autonomes et aux employés dont l'entreprise n'offre pas actuellement de régime de retraite enregistré.
\end{intro}
Les employeurs dont l'effectif dépasse un certain niveau sont tenus d'offrir la participation à un RVER s'ils n'offrent pas de régime de retraite enregistré ou de régime de retenues sur le salaire REER ou CELI.

Le contribuable établit sa propre cotisation au régime et l'employeur n'est pas tenu d'y participer. Toutefois l'employeur peut cotiser au RVER de son employé pourvu que celui-ci y participe également.

Contrairement aux régimes de retraite traditionnels, les RVER sont administrés par les institutions financières plutôt que par les employeurs, ce qui permet aux petites entreprises d'y participer sans exiger beaucoup de paperasse.

Le régime de pension agréé collectif (RPAC) est la contrepartie fédérale offerte aux provinces et territoires autres que le Québec.


\subsection{Traitement fiscal des cotisations à un RVER}
Les cotisations d'un employé à un RVER sont déductibles, mais pas sur la même ligne que le RPA. 

Toute contribution de l'employeur n'est pas imposable, mais elle diminue le montant que les contribuables peuvent cotiser à leur REER.

\href{https://www.rrq.gouv.qc.ca/fr/retraite/rver/Pages/rver.aspx}{Le régime volontaire d'épargne-retraite (RVER)}

\href{https://www.canada.ca/fr/agence-revenu/services/formulaires-publications/publications/t4040/reer-autres-regimes-enregistres-retraite.html}{REER et autres régimes enregistrés pour la retraite}, chapitre 8 - RPAC.



\section{Cotisations syndicales et professionnelles}
\begin{intro}
	Les contribuables qui paient des cotisations syndicales ou professionnelles peuvent bénéficier d'une déduction fédérale et d'un crédit d'impôt non remboursable au Québec pour les sommes qu'ils ont versées, pourvu que ces sommes se rapportent à leur emploi.
\end{intro}
\begin{note}
	Une caractéristique importante concernant la cotisation à un syndicat, à un ordre professionnel et à un comité paritaire:
	\begin{itemize}
		\item Une déduction est réclamée au fédéral;
		\item Un crédit d'impôt non remboursable est réclamé au Québec.
	\end{itemize}
\end{note}


\subsection{Cotisations admissibles}
Au fédéral et au Québec, les cotisations admissibles à la déduction et au crédit d'impôt comprennent notamment:
\begin{itemize}
	\item La cotisation annuelle versée à des associations professionnelles et syndicales. Les frais d'admission à un ordre professionnel ne sont pas déductibles, par exemple, les frais d'admission au Barreau, payés par un nouvel avocat;
	\item La cotisation versée à un office de professions, si le paiement est exigé par une loi provinciale ou territoriale;
	\item La cotisation obligatoire, incluant la prime d'une assurance responsabilité professionnelle, que le contribuable a versée pour lui permettre de conserver son statut professionnel reconnu par la loi; et
	\item La cotisation obligatoire effectuée à un comité paritaire ou consultatif (ou à un organisme semblable) ou à la Commission de la Construction du Québec (CCQ), comme l'exige la loi provinciale.
\end{itemize}

\subsubsection{Au Québec seulement}
Au Québec, les cotisations syndicales et professionnelles qui peuvent être réclamées comprennent également:
\begin{itemize}
	\item La cotisation à l'Association professionnelle des chauffeurs de taxi du Québec, pour permettre au contribuable de maintenir son permis de chauffeur de taxi;
	\item La cotisation annuelle à une association de salariés reconnue par le ministre du Revenu, ayant pour objets principaux l'étude, la sauvegarde et le développement des intérêts économiques de ses membres.
\end{itemize}

Si, pour un emploi donné, le contribuable réclame la cotisation à une association de salariés reconnue, il ne peut pas demander, pour cet emploi, le montant des cotisations effectuées à un syndicat, à un comité paritaire ou consultatif ou à un groupe semblable, à la Commission de la construction du Québec ou à l'Association professionnelle des chauffeurs de taxi du Québec.


\subsection{Traitement fiscal}
\subsubsection{Case 44 du T4 et Case F du RL-1}
Les cotisations syndicales payées par retenue sur le salaire sont indiquées à la case 44 du T4 et à la case F du Relevé 1. 

Pour la déclaration fédérale, les montants de la case 44 ou du reçu peuvent être entièrement déduits à la ligne 21200 de la T1. 

Au Québec, les montants (case F ou reçus) sont réclamés à titre de crédit d'impôt non remboursable à la ligne 397.1 du TP-1 au taux de 10~\% à la ligne 397.

\subsection{Assurance responsabilité professionnelle}
Au fédéral, ces primes peuvent être réclamées à la ligne 21200 de la T1. 

Cependant sur la TP-1, ces cotisations ne sont pas réclamées comme un crédit non remboursable, mais comme une dépense d'emploi à la ligne 207 de la TP-1, en utilisant le code 08 à la case 206.

\subsection{Remboursement de la TPS et de la TVQ}
Règle générale, la cotisation payée à une association professionnelle (excluant l'assurance responsabilité professionnelle) comprend la TPS et la TVQ. Si le contribuable veut réclamer le remboursement des taxes payées sur sa cotisation, il doit:

\begin{itemize}
	\item Au fédéral:
	\begin{itemize}
		\item Laisser le plein montant de la cotisation à la ligne 21200 de la T1;
		\item Remplir le formulaire suivant: GST 370 - Demande de remboursement de la TPS/TVH à l'intention des salariés et des associés;
		\item Demander le remboursement de la TPS à la ligne 45700 de la T1; et
		\item Le montant des deux taxes remboursées devient un montant imposable pour la prochaine année d'imposition. Le montant sera donc inscrit à la ligne 10400 de la T1.
	\end{itemize}
	\item Au Québec:
	\begin{itemize}
		\item Le montant inscrit à la ligne 397.1 doit exclure le montant payé pour la TPS et la TVQ;
		\item Remplir le formulaire suivant: VD-358 - Remboursement de la TVQ pour un salarié ou un membre d'une société de personnes; et
		\item Demander le remboursement de la TVQ à la ligne 459 de la TP-1.
		\item Le montant remboursé de la TVQ deviendra un montant imposable dans la prochaine année d'imposition, mais uniquement sur la déclaration fédérale. Le montant doit donc être inscrit à la ligne 10400 de la T1 de cette année.
	\end{itemize}
\end{itemize}

\href{https://www.canada.ca/content/dam/cra-arc/formspubs/pbg/gst370/gst370-22f.pdf}{Demande de remboursement de la TPS/TVH à l'intention des salariés et des associés}

\href{https://www.revenuquebec.ca/fr/services-en-ligne/formulaires-et-publications/details-courant/vd-358/}{Remboursement de la TVQ pour un salarié ou un membre d'une société de personnes}

Exemple: Cotisation professionnelle
\begin{itemize}
	\item T1
	\begin{description}
		\item[Ligne 21200] Cotisation annuelle + TPS + TVQ + Assurance responsabilité + Contribution Office des professions du Québec
	\end{description}
	\item TP-1
	\begin{description}
		\item[ligne 397.1] Cotisation annuelle + Contribution Office des professions du Québec
		\item[ligne 397] 10~\% de la ligne 397.1
		\item[ligne 207]  Assurance responsabilité + code 08 à la case 206
	\end{description}
\end{itemize}



\section{Dépenses d'emploi}
\begin{intro}
	Les employés qui répondent à certains critères stricts peuvent déduire à la ligne 22900 les dépenses déterminées engagées pour gagner un revenu d'emploi. Un règlement général s'applique à tous les employés et aux employés des vendeurs à commission. Des règles spéciales s'appliquent aux employés de professions spécifiques telles que le transport, la foresterie, la musique, l'art, les métiers, les apprentis mécaniciens et le ministère religieux.
\end{intro}

En vertu du règlement général, les employés ne peuvent déduire les dépenses d'emploi admissibles que s'ils remplissent toutes les conditions suivantes:
\begin{itemize}
	\item Leur contrat de travail les oblige à payer les frais;
	\item Un avantage non imposable n'a pas été et ne sera pas reçu pour les dépenses; et
	\item Ils ont le formulaire T2200 - Déclaration des conditions de travail qui a été signé et certifié par l'employeur.
\end{itemize}

Les dépenses qui peuvent être déductibles comprennent le coût des fournitures, les frais de déplacement, les frais d'automobile et certaines dépenses liées à l'entretien d'un bureau ou d'un espace de travail à domicile.

Même si certains employés peuvent déduire les dépenses d'emploi, la grande majorité ne le sont pas. Seuls les particuliers qui remplissent les conditions particulières énoncées dans la Loi de l'impôt sur le revenu peuvent déduire les dépenses d'emploi. Pour tous les autres, la seule compensation disponible dans le cadre du régime d'impôt sur le revenu est le montant canadien pour emploi, un crédit non remboursable.


\subsection{Remboursement du revenu d'emploi}
À l'occasion, les contribuables peuvent être tenus de rembourser certains types de revenus d'emploi qu'ils ont inclus dans leur revenu dans une déclaration de l'année en cours ou de l'année précédente. Par exemple, un employé peut avoir à rembourser les prestations d'un régime d'assurance-salaire à la suite d'une décision subséquente de la Commission des accidents du travail. Dans de tels cas, le montant du remboursement peut être déduit à la ligne 22900 de la déclaration de revenus dans l'année du remboursement.

\arcg{21}



\section{Exercice 3-2}
\setcounter{question}{0}
\begin{question}
	Pierre Tremblay est rédacteur technique pour la compagnie Microsoft. Il a reçu un T4 et un relevé 1 indiquant qu'il a payé 105~\$ de cotisations syndicales. Il a également un reçu officiel indiquant qu'il a versé un montant de 70~\$ à la Société des auteurs au cours de l'année. Il n'y a eu aucune TPS ni TVQ sur les cotisations.
\end{question}

\setcounter{sousQuestion}{0}
\begin{sousQuestion}
	Est-ce que Pierre peut réclamer les sommes payées au titre des cotisations syndicales et professionnelles sur ses T1 et TP-1? Expliquez votre réponse.
\end{sousQuestion}
Oui, Pierre peut réclamer le montant payé à titre de cotisations syndicales et professionnelles, car elles sont liées à son revenu d'emploi.

\begin{sousQuestion}
	Quel montant peut-il déduire au titre des cotisations syndicales, professionnelles? 
\end{sousQuestion}
Pierre peut déduire 175~\$ (105~\$ + 70~\$) sur sa déclaration fédérale seulement. Dans la déclaration du Québec, il peut demander un crédit non remboursable (et non une déduction).

\begin{sousQuestion}
	Auxquelles lignes des T1 et TP-1 peut-il réclamer les cotisations payées?
\end{sousQuestion}



\section{Remboursement de salaires ou de prestations d'assurance salaire}
\begin{intro}
	Les contribuables sont parfois obligés de rembourser une partie de leur salaire ou de leurs indemnités de remplacement de salaire. Par conséquent, ils peuvent demander une déduction.
\end{intro}

À l'occasion, il arrive qu'un contribuable doive rembourser certains revenus d'emploi qui sont déclarés dans l'année courante ou dans une année antérieure. 

Par exemple, un employé peut devoir rembourser ses prestations d'assurance salaire lorsqu'il s'est vu accorder une indemnité pour accidents du travail par la Commission des normes, de l'équité, de santé et de la sécurité au travail (CNESST).

Dans de telles situations, le montant sera déclaré dans la case 77 du T4 et dans la case A-3 ou à la case O-4 du RL-1.

Des déductions peuvent être réclamées à la ligne 22900 de la T1 et à la ligne 207 de la TP-1 avec le code 10 dans la case 206.

La déduction sur la déclaration de revenus doit avoir lieu dans l'année où le contribuable a remboursé le montant, même si le montant reçu en trop a été encaissé dans les années antérieures.



\section{Revenu net}
Les déductions de vos déclarations fédérales et québécoises peuvent ne pas toujours concorder, ce qui entraîne des montants de revenu net différents.



\section{Calcul du revenu imposable}
\begin{intro}
	Déductions plus spécifiques qui sont aussi reliées à l'emploi. Celles-ci ne concernent que quelques contribuables dans des situations particulières.
\end{intro}

Le revenu imposable peut être réduit par d'autres déductions plus spécifiques:
\begin{itemize}
	\item Pour le personnel des Forces canadiennes et des forces policières;
	\item Pour les options d'achat de titres;
	\item Pour les ristournes reçues d'une coopérative; et
	\item Pour un revenu d'emploi gagné sur un navire.
\end{itemize}



\section{Déductions spécifiques}
\subsection{Personnel des Forces armées et policières}
La déduction est indiquée à la case 43 du T4 et peut être réclamée à la ligne 24400 de la T1.

La déduction est indiquée à la case A-7 du Relevé 1 et peut être réclamée à la ligne 297, code \og 23 \fg{} de la case 296 de la TP-1.

Le montant de la case A-7 équivaut à la rémunération gagnée pendant le service à l'étranger, moins les cotisations à un RPA versé pendant cette période.


\subsection{Options d'achat de titres}
Un contribuable peut recevoir de son employeur une option d'achat de titres qui est une action du capital-actions de la société qui l'emploie. Le prix d'achat est déterminé d'avance. L'option devra être exercée dans un délai fixé.

Règle générale, l'employé a un certain temps pour exercer son option. Pendant ce temps, la valeur marchande des actions ou des unités peut fluctuer à la baisse ou à la hausse. Dans l'année où le contribuable exerce son option, si le prix de l'action ou de l'unité est moins élevé que sa valeur marchande au moment de l'exercice de l'option, la différence devient un avantage imposable pour l'employer. Heureusement, l'employé peut réclamer une \og Déduction pour options d'achat de titres \fg{}.

Dans la déclaration fédérale, l'avantage imposable est inclus à la case 14 du T4 du contribuable et inscrit dans la section \og Autres renseignements \fg{} où il est identifié par la case 38. Le montant de la déduction est inscrit à la case 39 ou 41 du même T4. Elle correspond à 50~\% de l'avantage imposable et est réclamée à la ligne 24900 de la T1.

Au Québec, l'avantage imposable est inscrit aux cases A et L du Relevé 1. La déduction (25~\% de l'avantage imposable) est inscrite aux cases L-9 ou L-10 du Relevé 1. La somme des entrées de toutes les cases L-9 et L-10 est réclamée à la ligne 297, code \og 02 \fg{} à la case 296, du TP-1.

Si le contribuable reçoit des options sur titres d'un emploi à l'extérieur du Québec, c'est-à-dire qu'aucun relevé 1 n'est reçu, une déduction de 25~\% du montant de la case 38 du T4 peut encore être réclamée à la ligne 297, code \og 02 \fg{} à la case 296, du TP-1. Une copie du T4 doit être soumise avec la déclaration.


\subsection{Ristourne reçue d'une coopérative}
Les ristournes versées à un membre d'une coopérative admissible sous forme de parts privilégiées sont inscrites à la case 030 du T4A et à la case O, code \og RL \fg{} à la case  \og Code (case O) \fg{} du Relevé 1. Le montant doit être déclaré à la ligne 13000 de la T1 et à la ligne 154, code \og 03 \fg{} à la case 153, de la TP-1.

Au Québec seulement, le contribuable peut réclamer une \og Déduction pour ristournes \fg{}. Le montant inscrit à la case O-2 du Relevé 1 peut être réclamé à la ligne 297, code \og 22 \fg{} de la case 296 du TP-1.


\subsection{Revenu d'emploi gagné sur un navire}
Au Québec seulement, un marin qui résidait au Québec en 2022 et qui possède une attestation du ministère des Transports du Québec peut bénéficier d'une déduction égale à 75~\% de la rémunération brute qu'un armateur admissible lui a versée dans l'année, pour une période où il a travaillé sur un navire affecté au transport international de marchandises. Il n'y a pas de déduction fédérale.

Si le propriétaire du navire a obtenu une attestation du Ministère des transports du Québec pour cet employé, le propriétaire doit inscrire le montant donnant droit à la déduction à la case A-6 du Relevé 1. Le contribuable peut demander la déduction à la ligne 297, code \og 08 \fg{} à la case 296, de la TP-1.


\subsection{Résumé des déductions spécifiques}
\begin{center}
	\begin{tabular}{|l|c|c|c|c|c|}
		\hline
		&  \textbf{T4}  &  \textbf{T1}   & \textbf{RL-1} & \multicolumn{2}{c|}{\textbf{TP-1}} \\ \cline{5-6}
		& \textbf{Case} & \textbf{Ligne} & \textbf{Case} & \textbf{Ligne} &   \textbf{Code}   \\ \hline
		Personnel des     &      43       &     24400      &      A-7      &      297       &        23         \\
		Forces armées     &               &                &               &                &     case 296      \\ \hline
		Options d'achat   &      39       &     24900      &      L-9      &      297       &        02         \\
		de titres         &      41       &                &     L-10      &                &     case 296      \\ \hline
		Ristourne reçue   &               &                &      O-2      &      297       &        22         \\
		d'une coopérative &               &                &               &                &     case 296      \\ \hline
		Marin             &               &                &      A-6      &      297       &        08         \\
		&               &                &               &                &     case 296      \\ \hline
	\end{tabular}
\end{center}

\rqg[s]{47-49}



\section{Revenu imposable}
Le revenu imposable du fédéral est différent de celui du Québec. Déductions qui peuvent être la cause de la différence:

\begin{itemize}
	\item La déduction de la cotisation syndicale ou professionnelle - au Fédéral seulement;
	\item La \og Déduction pour travailleur \fg{} - au Québec seulement;
	\item Options d'achat de titres (50~\% pour le fédéral et 25~\% pour le Québec); et
	\item Déduction pour revenu d'emploi gagné sur un navire - au Québec seulement.
\end{itemize}



\section{Exercice 3-3}
\setcounter{question}{0}
\begin{question}
	En janvier 2023, Mathieu Caron a obtenu une option d'achat de titres de la société qui l'emploie, ce qui lui a permis de faire l'acquisition de 250 actions au coût total de \numprint{2000}~\$. Au moment d'acheter les titres, leur juste valeur marchande était de 20~\$ par action. 
	
	Un avantage imposable de \numprint{3000}~\$ (250 x 12~\$) apparaît donc aux cases 38 et L de ses T4 et relevé 1 respectivement. 
	
	L'entreprise de son employeur n'est pas une PME poursuivant des activités innovantes. 
\end{question}
\setcounter{sousQuestion}{0}
\begin{sousQuestion}
	À quelles lignes de ses T1 et TP-1 Mathieu doit-il reporter l'avantage imposable?
\end{sousQuestion}
L'avantage imposable à la case 38 du T4 et à la case L du Relevé 1 est déjà inclus à la case 14 du T4 et à la case A du Relevé 1.

Les montants des cases 14 et A doivent être déclarés respectivement à la ligne 10100 de sa T1 et à la ligne 101 de sa déclaration TP-1.

\begin{sousQuestion}
	Quel est le pourcentage de l'avantage imposable que Mathieu peut réclamer comme déduction pour options d'achat de titres, au fédéral et au Québec?
\end{sousQuestion}
Au fédéral, Mathieu peut demander une déduction égale à 50~\% de l'avantage imposable qu'il a reçu.

Au Québec, il peut réclamer une déduction correspondant à 25~\% de l'avantage imposable qu'il a reçu.

\begin{sousQuestion}
	De quelle façon le montant de la déduction pour options d'achat de titres sera-t-il identifié sur ses feuillets de renseignements, au fédéral et au Québec?
\end{sousQuestion}
Au fédéral, le montant doit être inscrit à la case 39 ou 41 du T4. Au Québec, la déduction doit être inscrite soit à la case L-9, soit à la case L-10 du Relevé 1, selon le type d'options. 

\begin{sousQuestion}
	À quelle ligne de ses déclarations T1 et TP-1 peut-il réclamer une déduction pour options d'achat de titres?
\end{sousQuestion}
Au fédéral, à la ligne 24900 de sa T1.

Au Québec, il peut demander une déduction à la ligne 297 de sa TP1, en utilisant le code 02 à la case 296.
