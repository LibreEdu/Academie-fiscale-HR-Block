\chapter*{À propos de ce cours}
\addcontentsline{toc}{chapter}{À propos de ce cours}



\section*{Paragraphes d'introduction}
\addcontentsline{toc}{section}{Paragraphes d'introduction}
\begin{intro}
	Présente le sujet abordé 
\end{intro}



\section*{Notes et rappels}
\addcontentsline{toc}{section}{Notes}
\begin{note}
	Les notes et rappels constituent des informations importantes susceptibles de paraître dans les examens, il est donc judicieux d'y prêter une attention particulière.
	\begin{center}
		\begin{tabular}{l@{\qquad}l}
			Note   & \Large\righthand \\
			Rappel & \Large\aldine
		\end{tabular}
	\end{center}
\end{note}



\section*{Règles pour arrondir les nombres}
\addcontentsline{toc}{section}{Règles pour arrondir les nombres}
Soit $0,0xy$ la représentation des décimales d'un nombre. Si la troisième décimale est égale ou supérieure à 5 ($y \geq 5$) alors on incrémente la deuxième décimale ($x = x + 1$).

Exemple :
\begin{itemize}
	\item $0,014 \Rightarrow 0,01$
	\item $0,015 \Rightarrow 0,02$
\end{itemize}



\section*{Notation}
\addcontentsline{toc}{section}{Notation}
\begin{itemize}
	\item Devoir 1: 20~\% de la note finale
	\item Devoir 2: 20~\% de la note finale
	\item Examen théorique : 60~\% de la note finale
\end{itemize}
