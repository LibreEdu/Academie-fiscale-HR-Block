\chapter*{À propos de ce cours}
\addcontentsline{toc}{chapter}{À propos de ce cours}



\section*{Paragraphes d'introduction}
\addcontentsline{toc}{section}{Paragraphes d'introduction}
\begin{intro}
	Présente le sujet abordé 
\end{intro}



\section*{Notes et rappels}
\addcontentsline{toc}{section}{Notes}
\begin{note}
	Les notes et rappels constituent des informations importantes susceptibles de paraître dans les examens, il est donc judicieux d'y prêter une attention particulière.
	\begin{center}
		\begin{tabular}{l@{\qquad}l}
			Note   & \Large\righthand \\
			Rappel & \Large\aldine
		\end{tabular}
	\end{center}
\end{note}



\section*{Règles pour arrondir les nombres}
\addcontentsline{toc}{section}{Règles pour arrondir les nombres}
Soit $0,0xy$ la représentation des décimales d'un nombre. Si la troisième décimale est égale ou supérieure à 5 ($y \geq 5$) alors on incrémente la deuxième décimale ($x = x + 1$).

Exemple :
\begin{itemize}
	\item $0,014 \Rightarrow 0,01$
	\item $0,015 \Rightarrow 0,02$
\end{itemize}

\marginpar{\color{BrickRed}\Large{\danger}}
La somme d'arrondis \( (\text{arrondi}(x)+\text{arrondi}(y)) \) n’est pas forcément égale à l’arrondi d'une somme \( (\text{arrondi}(x+y) ) \).

Exemple:

\noindent
\begin{center}
	\begin{tblr}{
		width=0.8\linewidth,
		colspec = {X[c]|X[c]}
	}
		\SetCell[c=2]{c} x = 0,005 \qquad y = 0,005       &                                        \\ \hline
		arrondi(x) = 0,01                                 &                                        \\
		arrondi(y) = 0,01                                 & x $+$ y = 0,01                         \\
		arrondi(x) $+$ arrondi(y) = 0,0\textcolor{red}{2} & arrondi(x$+$y) = 0,0\textcolor{red}{1}
	\end{tblr}
\end{center}

Ainsi, si quelqu’un a un revenu de 1~\$ et un taux d’imposition de 15~\%, il a 0,15~\$ d’impôt. S’il a un crédit de 0,10~\$, cela fait une réduction d’impôt de 0,015~\$. On s’attend donc à une réduction 0,02~\$ en appliquant la règle d’arrondi. Son nouveau revenu est de 0,90~\$. Son impôt est de 0,135~\$, ce qui fait 0,14~\$ si on l’arrondit, soit une réduction de 0,01~\$ seulement.



\section*{Notation}
\addcontentsline{toc}{section}{Notation}
\begin{itemize}
	\item Devoir 1 (chapitre~6): 20~\% de la note finale
	\item Devoir 2 (chapitre~12): 20~\% de la note finale
	\item Examen théorique : 60~\% de la note finale
\end{itemize}
