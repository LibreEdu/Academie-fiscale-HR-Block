\chapter{Préliminaires}



\section{Introduction et objectifs}
\subsection{Introduction}
Ce chapitre est une introduction aux systèmes d'imposition canadien et québécois des revenus.


\subsection{Objectifs}
\begin{itemize}
	\item Expliquer comment le système de perception de l'impôt des particuliers a évolué jusqu'à aujourd'hui;
	\item Expliquer que c'est Revenu Québec qui a la responsabilité d'administrer la Loi sur les impôts du Québec et de percevoir l'impôt provincial à payer au gouvernement québécois;
	\item Déterminer qui doit produire une déclaration de revenus et qui a avantage à la faire même s'il n'y est pas légalement obligé;
	\item Identifier les types de revenus qui sont assujettis à l'impôt;
	\item Donner la signification des expressions \og Revenu total\fg{}, \og Revenu net\fg{} et \og Revenu imposable\fg{};
	\item Expliquer la date d'échéance et les trois méthodes disponibles pour déposer la déclaration d'impôt d'un particulier;
	\item Décrire les différents outils en ligne fournis par l'ARC et le RQ à l'usage des particuliers et des préparateurs de déclarations.
\end{itemize}


\subsection{Sujets du chapitre 1}
\begin{itemize}
	\item Évolution du système fiscal;
	\item Conditions de déclaration;
	\item Revenu assujetti à l'impôt;
	\item Introduction de déclarations des revenus;
	\item Avis de cotisation;
	\item Précision et tenue des registres.
\end{itemize}



\section{Système fiscal au Canada}
\begin{intro}
	Le système fiscal au Canada a subi plusieurs modifications au fil des années.  Il s'adapte aux besoins du temps et aux défis importants de la société en général.
\end{intro}
L'impôt sur le revenu est prélevé sur le \og revenu imposable\fg{} du contribuable qui a résidé au Canada à un moment donné au cours de l'année. Un \og contribuable\fg{}\ix{Contribuable} signifie une personne autre qu'une société (société par actions ou fiducie).

La perception des impôts est la responsabilité du gouvernement fédéral pour tous les provinces et territoires, sauf pour la province de Québec qui perçoit ses propres impôts. Les provinces et territoires appliquent leurs propres taux.



\section{Comment le système d'impôt canadien a-t-il évolué?}
\begin{description}
	\item[1867] Création du gouvernement canadien
	\item[1916] Impôt sur les sociétés: Loi taxant les profits d'affaires
	\item[1916] Loi de l'impôt de guerre sur le revenu
	\item[1927] Ministère du revenu national
	\item[1942] Introduction des déductions fiscales (retenues à la source)
	\item[\color{ForestGreen}1946] Henry et Leon Bloch fondent la United Business Company à Kansas City dans l'état du Missouri.
	\item[1949] Loi de l'impôt sur le revenu
	\item[1954] Deux déclarations pour les Québécois
	\item[\color{ForestGreen}1955] Henry and Richard Bloch crée H\&R Block, Inc.
	\item[1990] \acrfull{ted} au fédéral
	\item[1994] TED au Québec
	\item[2001] IMPÔTNET (fédéral)
\end{description}

Complément d'information: \href{https://www.hrblock.com/corporate/founders/}{It started with two brothers}

\begin{note}
	Le système fiscal du Québec est légèrement différent du système fiscal canadien, mais les dispositions de base sont les mêmes dans la loi fédérale sur l'impôt sur le revenu et la loi sur les impôts du Québec.
\end{note}



\section{Structure du système fiscal canadien}
\subsection{Régime d'autocotisation}
L'autocotisation\ix{Autocotisation} oblige les contribuables à calculer les impôts qu'ils doivent payer.


\subsection{Système d'imposition progressif}
\ix{Taux d'imposition}

\subsubsection{Taux d'imposition fédéraux (2023)}
\ca 
\begin{tabular}{|l|r|r|}
	\hline
	\textbf{Taux} & \multicolumn{2}{c|}{\textbf{$<$ Revenu $\leq$}} \\ \hline
	15~\%         &                      &      \numprint{53359}~\$ \\ \hline
	20,5~\%       &  \numprint{53359}~\$ &     \numprint{106717}~\$ \\ \hline
	26~\%         & \numprint{106717}~\$ &     \numprint{165430}~\$ \\ \hline
	29~\%         & \numprint{165430}~\$ &     \numprint{235675}~\$ \\ \hline
	33~\%         & \numprint{235675}~\$ &                          \\ \hline
\end{tabular}

\href{https://www.canada.ca/fr/agence-revenu/services/impot/particuliers/foire-questions-particuliers/taux-imposition-canadiens-particuliers-annee-courante-annees-passees.html}{Tranches de revenu imposable pour 2024}

\subsubsection{Taux d'imposition au Québec (2023)}
\qc
\begin{tabular}{|l|r|r|}
	\hline
	\textbf{Taux} & \multicolumn{2}{c|}{\textbf{$<$ Revenu $\leq$}} \\ \hline
	14~\%         &                      &      \numprint{49275}~\$ \\ \hline
	19~\%         &  \numprint{49275}~\$ &     \numprint{98540}~\$ \\ \hline
	24~\%         & \numprint{98540}~\$ &     \numprint{119910}~\$ \\ \hline
	25,75~\%      & \numprint{119910}~\$ &                          \\ \hline
\end{tabular}

\href{https://www.revenuquebec.ca/fr/citoyens/declaration-de-revenus/produire-votre-declaration-de-revenus/taux-dimposition/}{Tranches de revenu imposable pour 2024}
\url{}



\section{Résidence et assujettissement à l'impôt}
\begin{intro}
	Les revenus de toutes sources, à l'intérieur ou à l'extérieur du Canada, qu'ils soient perçus en espèces, sous forme de biens ou de services, sont imposables pour un résident du Canada, à moins qu'ils ne soient spécifiquement exonérés.
\end{intro}


\subsection{Assujettissement à l'impôt fédéral}
\begin{note}
	Au Canada, c'est la résidence et non pas la citoyenneté, qui détermine si le contribuable doit payer de l'impôt à l'ARC.
	
	Lorsqu'un contribuable réside au Canada durant toute l'année, son revenu mondial (revenus de toutes provenances) pour toute l'année est assujetti à l'impôt du Canada et à celui du Québec, sauf les revenus qui sont spécifiquement exonérés d'impôt par les lois de l'impôt sur le revenu.
\end{note}
\href{https://www.canada.ca/fr/agence-revenu/services/impot/impot-international-non-residents/renseignements-ont-deplaces/determination-votre-statut-residence.html}{Détermination de votre statut de résidence}


\subsection{Assujettissement à l'impôt provincial}
\begin{note}
	Selon l'ARC, un contribuable réside dans la province où il a son domicile et où réside sa famille, et non dans celle où il est physiquement présent le 31 décembre.
\end{note}


\subsection{Assujettissement à l'impôt du Québec}
\begin{note}
	Un contribuable est imposé dans la province où il réside le dernier jour de l'année d'imposition.
\end{note}
Un contribuable est imposé dans la province où il réside le dernier jour de l'année.

\href{https://www.revenuquebec.ca/fr/citoyens/votre-situation/statut-de-residence-et-assujettissement-a-limpot/liens-de-residence-consideres-dans-la-determination-du-statut-de-residence/}{Liens de résidence considérés dans la détermination du statut de résidence}



\section{Exercice 1-1}
\setcounter{question}{0}
\begin{question}
	L'Agence du revenu du Canada (ARC) perçoit les impôts fédéraux et provinciaux pour le contribuable résidant au Québec.
\end{question}
Faux. L'ARC perçoit seulement les impôts fédéraux au Québec. Revenu Québec perçoit les impôts provinciaux du contribuable résidant au Québec.

\begin{question}
	Quel est le facteur qui permet de déterminer si un particulier est assujetti à l'impôt du Canada: la citoyenneté ou la résidence?
\end{question}
Résidence canadienne.

\begin{question}
	Le Canada et le Québec possèdent tous les deux un système d'autocotisation. Qu'est-ce que cela signifie?
\end{question}
Cela signifie que les contribuables sont tenus de déterminer leur revenu imposable pour chaque année, de calculer leurs impôts à payer, puis de produire des déclarations de revenus pour déclarer ces montants aux gouvernements. Les contribuables utilisent les déclarations T1 et TP-1 pour déclarer leur impôt aux gouvernements fédéral et provincial, respectivement.

\begin{question}
	Pourquoi les systèmes d'imposition du Canada et du Québec sont-ils identifiés comme \og progressifs \fg{}?
\end{question}
Les systèmes fiscaux canadien et québécois sont \og progressifs\fg{} parce qu'ils imposent des taux d'imposition bas sur les revenus imposables faibles et modestes et que, à mesure que le revenu imposable augmente, le taux d'imposition augmente.

\begin{question}
	Paul Raymond résidait au Québec au 31 décembre de l'année d'imposition. Durant les huit premiers mois de l'année, il a vécu et travaillé en Alberta. Il a ensuite déménagé au Québec où il a travaillé le reste de l'année.
\end{question}
\setcounter{sousQuestion}{0}
\begin{sousQuestion}
	À quelle province Paul est-il assujetti à l'impôt provincial?
\end{sousQuestion}
Il doit payer l'impôt à la province du Québec.
\begin{sousQuestion}
	Paul a payé des impôts aux gouvernements du Canada, de l'Alberta et du Québec. Quelles déclarations de revenus doit-il produire?
\end{sousQuestion}
Il doit produire deux déclarations: une pour le gouvernement fédéral et une autre pour le gouvernement du Québec.

\begin{question}
	Suzanne Craig a travaillé au Québec et en Ontario. Durant l'année d'imposition, elle a établi sa résidence en Ontario.
\end{question}
\setcounter{sousQuestion}{0}
\begin{sousQuestion}
	À laquelle des provinces, Suzanne doit-elle payer un impôt provincial?
\end{sousQuestion}
Son impôt provincial est payable à l'Ontario.
\begin{sousQuestion}
	Des impôts ont été retenus à la source sur le salaire de Suzanne pour les gouvernements du fédéral, de l'Ontario et du Québec. Combien de déclarations doit-elle compléter et à qui doit-elle les envoyer?
\end{sousQuestion}
Elle doit produire une seule déclaration, la T1 fédérale qui est associée à la province de l'Ontario.



\section{Déclarations de revenus}
\begin{intro}
	Une déclaration de revenus est un formulaire de déclaration des revenus prescrit par l'ARC ou par Revenu Québec. Elle permet au contribuable de déclarer ses revenus et ses déductions et fournit les informations nécessaires à l'évaluation de l'impôt à payer. 
	
	La déclaration de revenus fédérale pour les particuliers s'appelle la Déclaration de revenus et de prestations (T1). Au Québec, elle s'appelle Déclaration de revenus (TP-1.D).
\end{intro}
\begin{note}
	Dans ces notes:
	\begin{description}
		\item[T1] Déclaration de revenus et de prestations fédérales\ix{T1}
		\item[TP-1] Déclaration de revenus du Québec\ix{TP-1}
	\end{description}
\end{note}
\href{https://www.canada.ca/fr/agence-revenu/services/formulaires-publications/trousses-impot-toutes-annees-imposition/trousse-generale-impot-prestations/quebec.html}{ARC -- Trousse d'impôt pour la province du Québec}

\href{https://www.revenuquebec.ca/fr/services-en-ligne/formulaires-et-publications/details-courant/tp-1/}{Déclaration de revenus, guide et annexes}



\section{Quelles sont les conditions de déclaration?}
\begin{intro}
	Produire ou ne pas produire sa déclaration de revenus? De nombreuses personnes choisissent de ne pas produire leur déclaration de revenus parce qu'elles pensent que ce n'est pas nécessaire ou parce qu'elles sont intimidées par le processus de production de la déclaration. Dans cette section, nous discutons de plusieurs raisons et avantages pour lesquels un particulier devrait produire une déclaration de revenus.
\end{intro}
Au fédéral et au Québec:
\caqc
\begin{itemize}
	\item Il a des impôts à payer à l'un ou l'autre des gouvernements;
	\item L'un ou l'autre des gouvernements lui a demandé de produire une déclaration;
	\item Il doit rembourser en totalité ou en partie la \acrfull{psv} ou les prestations d'\acrfull{ae} qu'il a reçues;
	\item Il a un solde du \acrfull{rap} ou du \acrfull{reep} qui n'a pas été remboursé;
	\item Il veut réclamer un remboursement d'impôt;
	\item Il produit une déclaration afin d'inclure un revenu admissible dans le calcul de son maximum déductible au titre des \acrshort{reer} et de mettre à jour ce maximum;
	\item Il veut recevoir le crédit pour la \acrshort{tps};
	\item Il veut faire une demande de renouvellement du \acrfull{srg} ou de l'Allocation (au conjoint);
	\item Il veut commencer ou continuer à recevoir l'\acrfull{ace} (dans le cas d'un couple, le contribuable et son conjoint doivent tous les deux produire une déclaration de revenus au fédéral);
	\item Il veut transférer les frais de scolarité à un parent ou à un grand-parent;
	\item Il veut reporter la partie inutilisée de ses frais de scolarité, du montant relatif aux études et du montant pour manuels;
	\item Il a vendu sa résidence principale;
	\item Il a reçu ou désire recevoir des versements anticipés de l'\acrfull{act};
	\item Il veut déclarer un revenu qui lui permettrait d'augmenter sa limite de crédits de formation au Canada.
\end{itemize}

Au Québec:
\qc
\begin{itemize}
	\item Il doit payer des cotisations au \acrfull{rrq}, au \acrfull{fss} et au \acrfull{rqap};
	\item Il doit payer une cotisation au \acrfull{ramq};
	\item Il a reçu ou désire recevoir des versements anticipés du Crédit d'impôt pour frais de garde d'enfants, du Crédit d'impôt relatif à la prime au travail, du Crédit d'impôt pour le maintien à domicile d'une personne âgée et du Crédit d'impôt pour traitement de l'infertilité;
	\item Il désire demander le Crédit pour frais de garde d'enfants, les Crédits d'impôt relatifs à la prime au travail (prime au travail, prime au travail adaptée, supplément pour prestataire quittant l'assistance sociale), le Crédit pour maintien à domicile d'une personne âgée;
	\item Il désire demander la subvention pour aînés relative à une hausse de taxes municipales;
	\item Il veut recevoir ou maintenir les paiements de l'Allocation famille (dans le cas d'un couple, tous les deux doivent produire une déclaration du Québec);
	\item Il désire transférer à son conjoint la partie inutilisée de ses crédits d'impôt non remboursables pour permettre à ce dernier de réduire son impôt (exige que les deux conjoints produisent une déclaration);
	\item Il désire demander le Crédit d'impôt pour solidarité;
	\item Il désire demander les autres crédits ou remboursements de la ligne 462 de la déclaration provinciale du Québec.
\end{itemize}

Même si aucune des circonstances ci-dessus ne s'applique, un contribuable peut tout de même bénéficier de la production d'une déclaration de revenus.
\arcg{7}
\rqg{3}



\section{Exercice 1-2}
\setcounter{question}{0}
\begin{question}
	Énumérez cinq circonstances qui obligent les contribuables à produire une déclaration de revenus.
\end{question}
Les cinq circonstances sont:
\begin{enumerate}
	\item Ils doivent de l'impôt au gouvernement fédéral ou provincial;
	\item Le gouvernement provincial ou fédéral leur a demandé de produire une déclaration;
	\item Ils doivent rembourser une partie ou la totalité des \acrfull{psv} ou de l'\acrfull{ae} qu'ils ont reçues;
	\item Ils sont tenus de faire un remboursement dans le cadre du \acrfull{rap} ou du \acrfull{reep};
	\item Ils veulent réclamer un remboursement d'impôt.
\end{enumerate}

\begin{question}
	Nommez trois situations où le contribuable et son conjoint doivent tous les deux produire une déclaration du Québec.
\end{question}
Voici les trois situations:
\begin{enumerate}
	\item Recevoir les paiements de l'Allocation famille versés par Retraite Québec;
	\item Transférer au conjoint la partie inutilisée de ses crédits d'impôt non remboursables;
	\item Réclamer le crédit d'impôt pour solidarité.
\end{enumerate}

\begin{question}
	Quelle est la date limite de 2024 pour les contribuables non travailleurs autonomes pour produire leur déclaration de revenus?
\end{question}
La date limite est le mardi 30 avril. Si ca tombait le week-end, elle serait repoussée au lundi suivant.

\begin{question}
	Dans un couple, l'un des conjoints est un employé salarié et l'autre est un travailleur autonome.
\end{question}
\setcounter{sousQuestion}{0}
\begin{sousQuestion}
	Quelle est la date limite pour produire leurs déclarations de revenus pour l'année d'imposition 2024?
\end{sousQuestion}
Comme l'un des conjoints est travailleur autonome, les deux conjoints peuvent attendre jusqu'au 15 juin 2024 pour produire leur déclaration de revenus.
\begin{sousQuestion}
	En supposant qu'ils ont un solde dû, quelle est la date limite pour le payer s'ils veulent éviter de payer des intérêts?
\end{sousQuestion}
Les conjoints ont jusqu'au 30 avril 2024 pour payer tout solde dû, même s'ils ont le droit de produire leurs déclarations jusqu'au 15 juin 2024.

\begin{question}
	Quelle est l'année d'imposition d'un particulier?
\end{question}
L'année d'imposition s'étend du 1\up{er} janvier au 31 décembre.



\section{Revenus assujettis à l'impôt}
\begin{intro}
	Les revenus de toutes provenances, incluant ceux du Canada et hors Canada, sont imposables, à moins d'être spécifiquement exemptés. Les revenus peuvent être reçus en espèces, en biens ou en services. Certains revenus pourraient être exemptés d'impôts. Le contribuable canadien qui les reçoit doit obligatoirement les déclarer.
\end{intro}


\subsection{Qu'est-ce qu'un revenu?}
\caqc
\begin{itemize}
	\item Revenus provenant d'une charge ou d'un emploi;
	\item Revenus provenant d'une entreprise ou de biens;
	\item Gains en capital;
	\item Autres sources de revenus (par exemple, les prestations d'un régime de retraite).
\end{itemize}

Ne pas inclure:
\caqc
\begin{itemize}
	\item Le crédit pour la TPS versée par le gouvernement fédéral et le crédit d'impôt pour la solidarité versée par Revenu Québec;
	\item L'\acrfull{ace} et la \acrfull{peh} versées par le gouvernement fédéral;
	\item L'Allocation famille, le supplément pour enfants handicapés, le supplément pour l'achat de fournitures scolaires et le supplément pour enfant handicapé nécessitant des soins exceptionnels versés par Retraite Québec;
	\item Les indemnités de grève;
	\item Au fédéral, les indemnités reçues d'une province ou d'un territoire pour compenser les victimes d'actes criminels ou d'accidents d'automobile et les indemnités versées par la Société de l'assurance automobile du Québec;
	\item Les gains de loterie;
	\item La plupart des cadeaux et des biens reçus en héritage;
	\item Les montants reçus en raison d'une invalidité ou du décès d'un ancien combattant résultant de sa participation à la guerre;
	\item Le produit d'une police d'assurance-vie reçue à la suite d'un décès;
	\item Les prestations reçues d'un régime d'assurance salaire ou d'assurance revenu si votre employeur n'a pas cotisé à ce régime;
	\item L'allocation reçue dans le cadre du programme Allocation-logement.
	\item Le programme Allocation-logement est une aide financière conçue par le gouvernement du Québec. Cette aide est destinée aux familles avec enfants et aux personnes âgées à faible revenu qui consacrent une part trop importante de leur budget au paiement du loyer. L'aide reçue n'est pas imposable;
	\item Les crédits d'impôt relatifs à l'allocation canadienne pour les travailleurs  versés par le gouvernement fédéral;
	\item Les crédits d'impôt relatifs à la prime au travail versée par Revenu Québec;
	\item Au fédéral, les bourses scolaires primaires et secondaires et les bourses d'études reçues pour un étudiant admissible;
	\item La plupart des montants reçus d'un \acrfull{celi}.
\end{itemize}
\begin{note}
	Les revenus gagnés sur l'un des montants ci-dessus (tels que les intérêts gagnés lors de l'investissement des gains de loterie) sont \textbf{imposables}.
\end{note}


\subsection{Montants compris dans le revenu, mais non imposables}
\begin{itemize}
	\item Les indemnités pour accident du travail;
	\item Le supplément de revenu garanti et l'allocation au conjoint versés par le fédéral;
	\item Les revenus exonérés d'impôt en vertu d'une convention fiscale;
	\item Au fédéral, les prestations d'assistance sociale et autres prestations semblables;
	\item Au Québec, les indemnités de la \acrfull{saaq} et les indemnités pour victimes d'actes criminels;
	\item Au Québec, les bourses d'études ou toute aide semblable.
\end{itemize}
\begin{note}
	Même si ces revenus ne sont pas imposables, il est important de les déclarer afin qu'ils soient inclus dans le revenu net.
\end{note}
\arcg{11}
\rqg{19}



\section{Exercice 1-3}
\setcounter{question}{0}
\begin{question}
	Énumérez cinq types de versements qui ne sont jamais inclus dans la déclaration de revenus.
\end{question}
\begin{itemize}
	\item Le crédit pour la TPS versé par le gouvernement fédéral et le crédit d'impôt pour la solidarité versé par Revenu Québec;
	\item L'allocation canadienne pour enfants et la prestation pour enfants handicapés versées par le gouvernement fédéral;
	\item L'allocation famille et le supplément pour enfants handicapés versés par Retraite Québec;
	\item Les indemnités de grève;
	\item Les gains de loterie
\end{itemize}

\begin{question}
	Qu'est-ce que le revenu net? Pourquoi ce montant est-il important?
\end{question}
Le \og revenu net\fg{} est le total des revenus de toutes provenances, moins les déductions spécifiques admissibles. Il est utilisé pour déterminer l'admissibilité du contribuable à plusieurs crédits d'impôt et prestations sociales.

\begin{question}
	Qu'est-ce qu'un \og revenu imposable\fg{}? Et pourquoi est-ce important?
\end{question}
Le revenu imposable correspond au revenu net moins certaines déductions autorisées par la Loi de l'impôt sur le revenu. C'est important car ce sont les revenus sur lesquels l'impôt est prélevé.

\begin{question}
	Énumérez quatre types de revenus non imposables qui sont inclus dans le revenu net, mais qui ne sont pas inclus dans le calcul du revenu imposable.
\end{question}
\begin{enumerate}
	\item Les indemnités pour accidents de travail;
	\item Le supplément de revenu garanti et l'allocation au conjoint versés par le fédéral;
	\item Les prestations d'assistance sociale (au fédéral seulement);
	\item Au Québec seulement, les bourses d'études ou toute aide financière semblable.
\end{enumerate}



\section{Les parties principales des T1 et TP-1}
\begin{intro}
	Le processus de calcul de l'impôt est très similaire pour l'impôt fédéral et l'impôt du Québec. La T1 et la TP-1 sont utilisées pour calculer l'impôt dû par un particulier.
\end{intro}
T1:
\ix{T1}
\ca
\begin{description}
	\item[Étape 1] Identification et autres renseignements;
	\item[Étape 2] Revenu total (15000);
	\item[Étape 3] Revenu net (23600);
	\item[Étape 4] Revenu imposable (26000);
	\item[Étape 5] Impôt fédéral;
	\begin{description}
		\item[Partie A] Impôt fédéral sur le revenu imposable
		\item[Partie B] Crédits d'impôt non remboursables fédéraux (35000)
		\item[Partie C] Impôt fédéral net (42000)
	\end{description}
	\item[Étape 6] Remboursement (48400) ou solde dû (48500).
\end{description}

TP-1:
\ix{TP-1}
\qc
\begin{itemize}
	\item Renseignements sur vous;
	\item Renseignements sur votre conjoint au 31 décembre;
	\item Revenu total (199);
	\item Revenu net (275);
	\item Revenu imposable (299);
	\item Crédits d'impôt non remboursables (399);
	\item Impôt et cotisations (450);
	\item Remboursement ou solde à payer (479).
\end{itemize}


\subsection{Revenu total}
\ix{Revenu total}
Ce sont les revenus de toutes provenances, canadiennes et étrangères.


\subsection{Revenu net}
\ix{Revenu net}
C'est le revenu total moins certaines déductions, il sert à déterminer l'admissibilité du contribuable aux crédits d'impôt et aux prestations sociales.


\subsection{Revenu imposable}
\ix{Revenu imposable}
C'est le revenu total moins toutes les déductions, il sert de base au calcul du solde dû.


\subsection{Crédits d'impôt non remboursables}
\ix{Crédits d'impôt non remboursables}
Ils servent à réduire le montant de l'impôt à payer. Si le montant total dépasse l'impôt calculé, l'excédent n'est pas remboursé.


\subsection{Impôt et cotisations}
Au fédéral, le calcul se fait aux pages 7 et 8 de la déclaration T1.

Au Québec, c'est aux pages 3 et 4 de la TP-1. C'est là également que sont calculés les cotisations:
\begin{itemize}
	\item au \acrfull{rqap};
	\item au \acrfull{fss};
	\item au \acrfull{ramq}.
\end{itemize}


\subsection{Remboursement ou solde dû}
Une fois les montants du \og Total de l'impôt à payer\fg{} \cat{} et de l'\og Impôt et cotisations \fg{}~\qct{} calculés, le contribuable doit déterminer tous les crédits remboursables auxquels il est admissible.



\section{Saisir les données correctement}
\begin{intro}
	Assurez-vous que vous êtes attentif aux détails. L'impact d'une erreur typographique, qu'il s'agisse d'une lettre dans un nom ou d'un chiffre manquant dans un montant, pourrait vous causer beaucoup de difficultés. Ouvrez l'œil.
\end{intro}
Produire une \og déclaration complète\fg{} signifie déclarer tous les revenus que la législation oblige à déclarer.

Le contribuable a l'obligation de déclarer tous les revenus qu'il perçoit.

Produire une déclaration exacte lorsqu'il s'agit d'une déclaration sur papier signifie que toutes les informations doivent être saisies de manière précise, lisible et sur les bonnes lignes.

\ix{T1}
Les points noirs, à la droite de certaines lignes de la T1, indiquent que le montant doit provenir d'un calcul effectué sur une annexe, un formulaire quelconque ou une grille de calcul.



\section{Exercice 1-4}
\setcounter{question}{0}
\begin{question}
	Identifier les principales sections de la déclaration de revenus fédérale, Déclaration de revenus et de prestations.
\end{question}
\begin{description}
	\item[Étape 1] Identification et autres renseignements
	\item[Étape 2] Revenu total
	\item[Étape 3] Revenu net
	\item[Étape 4] Revenu imposable
	\item[Étape 5] Impôt fédéral
	\item[Étape 6] Remboursement ou solde dû
\end{description}

\begin{question}
	Identifier les principales sections de la déclaration de revenus du Québec.
\end{question}
\begin{itemize}
	\item Renseignements sur vous et  sur votre conjoint au 31 décembre 2023
	\item Revenu total
	\item Revenu net
	\item Revenu imposable
	\item Crédits non remboursables
	\item Impôt et cotisations
	\item Remboursement ou solde à payer
\end{itemize}

\begin{question}
	Que signifie l'expression \og Produire une déclaration complète et exacte\fg{}?
\end{question}
Produire une déclaration complète signifie déclarer tous les revenus qui doivent être déclarés. Remplir une déclaration exacte signifie inscrire toutes les informations de manière précise, lisible et aux lignes appropriées.



\section{Comment pouvons-nous déposer?}
\begin{intro}
	Une fois que les deux déclarations ont été correctement préparées, elles doivent être produites, c'est-à-dire envoyer à l'ARC et à Revenu Québec.
\end{intro}


\subsection{Il existe plusieurs façons de produire une déclaration}
\subsubsection{Production papier}
Un couple doit poster deux enveloppes. Si une personne produit des déclarations pour \textbf{plusieurs années, toutes les déclarations} de cette personne peuvent se trouver dans la \textbf{même enveloppe}.

\subsubsection{Transmission électronique}
\setcounter{annee}{2023}
L'ARC permet la production électronique pour l'année en cours et les six années précédentes (\addtocounter{annee}{-6}\theannee{} $\rightarrow$ \addtocounter{annee}{6}\theannee{}).

\setcounter{annee}{2023}
Revenu Québec permet la production électronique pour l'année en cours et les trois années précédentes (\addtocounter{annee}{-3}\theannee{} $\rightarrow$ \addtocounter{annee}{3}\theannee{}).

\paragraph{TED}
La \acrshort{ted} est un système qui permet aux fournisseurs de services de production électronique enregistrés d'envoyer par voie électronique des renseignements sur les déclarations de revenus des particuliers à l'ARC et à RQ.

\cat\href{https://www.canada.ca/fr/agence-revenu/services/formulaires-publications/formulaires/t183.html}{T183 Déclaration de renseignements pour la transmission électronique d'une déclaration de revenus et de prestations d'un particulier}

\qct\href{https://www.revenuquebec.ca/fr/services-en-ligne/formulaires-et-publications/details-courant/tp-1000-te/}{TP-1000.TE Transmission par Internet de la déclaration de revenus d'un particulier}

\paragraph{IMPÔTNET}
\ix{IMPÔTNET}
IMPÔTNET permet aux particuliers d'envoyer leurs propres déclarations à l'ARC  et à Revenu Québec en ligne.

\subsubsection{Déclarer simplement par téléphone}
Un nombre limité de contribuables peuvent produire leur déclaration de revenus par téléphone.


\subsection{Dates limites pour la production des déclarations}
\ix{Date limite}
La date limite est le 30 avril. Le travailleur autonome a jusqu'au 15 juin, incluant son conjoint. Par contre, si le contribuable a un solde dû, ce solde doit être payé au plus tard le 30 avril. Si la date limite de production est un samedi, un dimanche ou un jour férié, le contribuable doit produire sa déclaration au plus tard le jour ouvrable suivant.

\subsubsection{Date limite 2024}
\noindent
\begin{tabular}[t]{cc}
	\raisebox{13ex}{
		\begin{minipage}[t]{.5\textwidth}
			La date limite du 30 avril est un mardi en 2024, la date limite reste le 30~avril.
		\end{minipage}
	}
	&
	\begin{minipage}[t]{.3\textwidth}
		% https://tikz.dev/library-calender
		\tikz[every day/.style={anchor=mid}]
		\calendar
		[dates=2024-04-01 to 2024-04-30,
		week list,
		month label above centered,
		month text=\%mt \%y-]
		if (weekend) [red]
		if (equals=2024-04-30) {\draw (0,0) circle (8pt);};
	\end{minipage}
	\\
\end{tabular}



\section{Exercice 1-5}
\setcounter{question}{0}
\begin{question}
	Quels sont les avantages de Dépôt électronique?
\end{question}
Il est plus rapide, il y a moins de retards de traitement, moins d'erreurs, il n'y a pas de frais d'impression (ou d'autres frais liés à la production d'une déclaration sur papier) et il n'y a pas de frais d'affranchissement.

\begin{question}
	Expliquer les termes TED et IMPÔTNET?
\end{question}
Les deux services permettent de soumettre une déclaration de revenus par voie électronique à l'ARC et à Revenu Québec. La différence réside dans le fait que:
\begin{itemize}
	\item La TED exige qu'un fournisseur de services électroniques enregistré prépare et soumette la déclaration de revenus au nom du contribuable. 
	\item Avec IMPÔTNET, les contribuables peuvent préparer et envoyer leurs propres déclarations de revenus à l'aide d'un logiciel fiscal certifié et d'Internet. 
\end{itemize}



\section{Outils en ligne de l'ARC}
\begin{intro}
	L'\acrshort{arc} fournit plusieurs services électroniques aux particuliers et aux préparateurs de déclarations de revenus afin d'améliorer l'efficacité et l'exactitude lors de la préparation de déclarations de revenus.
\end{intro}


\subsection{Mon dossier}
\begin{note}
	L'ARC n'enverra les codes que par la poste et ne les donnera jamais par téléphone.
\end{note}
\href{https://www.canada.ca/fr/agence-revenu/services/services-electroniques/services-numeriques-particuliers/dossier-particuliers.html}{Mon dossier pour les particuliers}


\subsection{Applications mobiles de l'ARC}
\subsubsection{MonARC}
C'est une application qui permet de consulter les renseignements fiscaux clés.

\subsubsection{MesPrestations ARC}
C'est une application qui permet de consulter les informations sur les prestations.

\href{https://www.canada.ca/fr/agence-revenu/services/services-electroniques/applications-mobiles-arc.html}{Applications mobiles – Agence du revenu du Canada}


\subsection{Représenter un client}
Cela permet aux représentants d'accéder aux informations fiscales.

\href{https://www.canada.ca/fr/agence-revenu/services/services-electroniques/representer-client.html}{Représenter un client}


\subsection{Préremplir ma déclaration (PRD)}
Cela permet de remplir automatiquement des parties d'une déclaration de revenus.

\href{https://www.canada.ca/fr/agence-revenu/services/services-electroniques/a-propos-preremplir-declaration.html}{Préremplir ma déclaration}



\section{Outils en ligne de RQ}
\begin{intro}
	\acrfull{rq} offre également une suite de services électroniques pour les particuliers et les préparateurs de déclarations afin d'améliorer l'efficacité et l'exactitude lors de la préparation de déclarations de revenus. La portée de ces services n'est pas aussi vaste que celle de l'ARC.
\end{intro}


\subsection{Mon dossier}
C'est un portail qui permet de consulter ses propres informations.

\href{https://www.revenuquebec.ca/fr/services-en-ligne/services-en-ligne/services-en-ligne/citoyens/}{Tous les services en ligne – Citoyens}


\subsection{Représentants PRO +}
Un particulier peut accorder une autorisation à une personne désignée pour être son représentant lorsqu'il communique avec Revenu Québec.

\href{https://www.revenuquebec.ca/fr/services-en-ligne/services-en-ligne/services-en-ligne/representants-professionnels/}{Tous les services en ligne – Représentants professionnels}


\subsection{Téléchargement de données fiscales}
Cela permet de remplir automatiquement des parties d'une déclaration de revenus.



\section{Qu'est-ce qu'un avis de cotisation?}
\begin{intro}
	L'avis de cotisation résume le calcul du remboursement ou du solde dû.
\end{intro}
\begin{note}
	Si un contribuable s'inscrit au courrier en ligne, tous les avis de cotisation ou de nouvelle cotisation émis après l'inscription peuvent être consultés sous Mon dossier sur le site Web de l'ARC ou RQ, et aucune copie papier n'est envoyée par la poste.
\end{note}


\subsection{Avis de cotisation express}
ADC express permet de voir l'\acrfull{adc} immédiatement après que la déclaration a été reçue et traitée par l'ARC.

\href{https://www.canada.ca/fr/agence-revenu/services/services-electroniques/a-propos-express.html}{ADC express}



\section{Registres et reçus}
\begin{intro}
	Qu'une déclaration soit produite sur papier ou par voie électronique, la loi relative à l'impôt sur le revenu exige de tous les contribuables qu'ils tiennent des registres appropriés pour déterminer le montant de l'impôt à payer.
\end{intro}
Les documents fiscaux doivent être conservés pendant une période d'au moins six ans. Le gouvernement \cat \qct est présumé avoir raison, à moins que le contribuable puisse leur prouver le contraire.
\arcg{32}


\subsection{Pièces justificatives non requises – Québec}
Au Québec, le contribuable n'a plus à joindre ses relevés et ses justificatifs. S'il a gagné un revenu hors Québec, il doit joindre à sa TP-1 les feuillets T4 du fédéral.
\rqg{8}



\section{Exercice 1-6}
\setcounter{question}{0}
\begin{question}
	Quels sont les documents que le contribuable doit conserver? Pendant combien de temps doivent-ils être conservés?
\end{question}
Les contribuables doivent tenir un registre adéquat pour déterminer le montant des impôts qu'ils doivent payer. Tous les reçus et autres documents fiscaux justificatifs doivent être conservés pendant au moins six ans après l'année fiscale au cours de laquelle ils ont été déclarés. Si un contribuable est en retard pour déposer sa déclaration, il doit également conserver ses documents pendant au moins six ans après la date à laquelle il a déposé sa déclaration.

\begin{question}
	Qu'est-ce qu'un avis d'imposition?
\end{question}
L'ARC et Revenu Québec émettent chacun un avis de cotisation après qu'un contribuable a produit une déclaration de revenus. Ces avis résument le calcul du remboursement du contribuable ou de l'impôt à payer.



\section{Sommaire du chapitre 1}
\begin{itemize}
	\item Le gouvernement fédéral et le gouvernement du Québec prélèvent des impôts sur le revenu.
	\item L'impôt fédéral et québécois est régi par la Loi de l'impôt sur le revenu.
	\item L'Agence du revenu du Canada (ARC) perçoit les impôts au nom du gouvernement fédéral et de toutes les provinces, à l'exception du Québec.
	\item Les Canadiens sont imposés en fonction de leur lieu de résidence et non de leur citoyenneté.
	\item Les résidents canadiens sont imposables sur leur revenu de toutes provenances pour toute l'année.
	\item L'année d'imposition des particuliers est l'année civile, soit du 1er janvier au 31 décembre.
	\item La date limite pour les déclarations de revenus des particuliers est le 30 avril de l'année suivante (reportée au lundi suivant lorsque le 30 avril est un samedi ou un dimanche). La date d'échéance est reportée au 15 juin si le contribuable ou son époux ou conjoint de fait est un travailleur autonome.
	\item Le revenu net est le montant utilisé pour déterminer l'admissibilité à de nombreux crédits d'impôt et prestations sociales.
	\item Le revenu imposable est le revenu sur lequel l'impôt est prélevé.
	\item Les impôts fédéraux déclarés sur une T1 pour les particuliers sont calculés à l'étape 5 de la Déclaration de revenus et de prestations.
	\item Au Québec, les impôts et crédits provinciaux sont déclarés sur TP-1, qui est produit auprès de Revenu Québec.
	\item Les déclarations peuvent être soumises à l'ARC et à Revenu Québec en format papier (par la poste) ou par voie électronique (TED ou IMPÔTNET). Dans des situations limitées, il est également possible de soumettre une déclaration de revenus par téléphone.
	\item Les particuliers et les préparateurs de déclarations de revenus peuvent utiliser divers outils en ligne. Pour le gouvernement fédéral, les outils sont Mon dossier, MonARC, Mes prestations de l'ARC, Représenter un client et Préremplir ma déclaration. Québec, Mon dossier, Rep+ PRO et Téléchargement de données fiscales.
\end{itemize}
